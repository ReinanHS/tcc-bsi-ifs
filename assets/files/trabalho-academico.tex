%%%%%%%%%%%%%%%%%%%%%%%%%5
%% abtex2-modelo-trabalho-academico.tex, v<VERSION> laurocesar
%% Copyright 2012-<COPYRIGHT_YEAR> by abnTeX2 group at http://www.abntex.net.br/
%%
%% This work may be distributed and/or modified under the
%% conditions of the LaTeX Project Public License, either version 1.3
%% of this license or (at your option) any later version.
%% The latest version of this license is in
%%   http://www.latex-project.org/lppl.txt
%% and version 1.3 or later is part of all distributions of LaTeX
%% version 2005/12/01 or later.
%%
%% This work has the LPPL maintenance status `maintained'.
%%
%% The Current Maintainer of this work is the abnTeX2 team, led
%% by Lauro César Araujo. Further information are available on
%% http://www.abntex.net.br/
%%
%% This work consists of the files abntex2-modelo-trabalho-academico.tex,
%% abntex2-modelo-include-comandos and abntex2-modelo-references.bib
%%

% ------------------------------------------------------------------------
% ------------------------------------------------------------------------
% abnTeX2: Modelo de Trabalho Academico (tese de doutorado, dissertacao de
% mestrado e trabalhos monograficos em geral) em conformidade com
% ABNT NBR 14724:2011: Informacao e documentacao - Trabalhos academicos -
% Apresentacao
% ------------------------------------------------------------------------
% ------------------------------------------------------------------------

\documentclass[
	% -- opções da classe memoir --
	12pt,				% tamanho da fonte
%  openright,			% capítulos começam em pág ímpar (insere página vazia caso preciso)
	oneside,			% para impressão em recto e verso use twoside
	a4paper,			% tamanho do papel.
	% -- opções da classe abntex2 --
	%chapter=TITLE,		% títulos de capítulos convertidos em letras maiúsculas
	%section=TITLE,		% títulos de seções convertidos em letras maiúsculas
	%subsection=TITLE,	% títulos de subseções convertidos em letras maiúsculas
	%subsubsection=TITLE,% títulos de subsubseções convertidos em letras maiúsculas
	% -- opções do pacote babel --
	english,			% idioma adicional para hifenização
	french,				% idioma adicional para hifenização
	spanish,			% idioma adicional para hifenização
	brazil				% o último idioma é o principal do documento
	]{abntex2}

% ---
% Pacotes básicos
% ---
\usepackage{times}			    % Usa fonte times
\renewcommand{\ABNTEXchapterfont}{\normalfont} % para aplicar a fonte escolhida em tudo
\usepackage[T1]{fontenc}		% Selecao de codigos de fonte.
\usepackage[utf8]{inputenc}		% Codificacao do documento (conversão automática dos acentos)
\usepackage{lastpage}			% Usado pela Ficha catalográfica
\usepackage{indentfirst}		% Indenta o primeiro parágrafo de cada seção.
\usepackage{color}				% Controle das cores
\usepackage{graphicx}			% Inclusão de gráficos
\usepackage{microtype} 			% para melhorias de justificação
% ---

% ---
% Pacotes adicionais, usados apenas no âmbito do Modelo Canônico do abnteX2
% ---
\usepackage{lipsum}				% para geração de dummy text
% ---

% ---
% Pacotes de citações
% ---
\usepackage[brazilian,hyperpageref]{backref}	 % Paginas com as citações na bibl
\usepackage[alf]{abntex2cite}	% Citações padrão ABNT

% ---
% CONFIGURAÇÕES DE PACOTES
% ---

% ---
% Configurações do pacote backref
% Usado sem a opção hyperpageref de backref
\renewcommand{\backrefpagesname}{Citado na(s) página(s):~}
% Texto padrão antes do número das páginas
\renewcommand{\backref}{}
% Define os textos da citação
\renewcommand*{\backrefalt}[4]{
	\ifcase #1 %
		Nenhuma citação no texto.%
	\or
		Citado na página #2.%
	\else
		Citado #1 vezes nas páginas #2.%
	\fi}%
% ---

% ---
% Informações de dados para CAPA e FOLHA DE ROSTO
% ---
\titulo{O uso e avaliação de um sistema de e-learning no ensino de linguagens de
programação}
\autor{Reinan Gabriel Dos Santos Souza}
\data{2022}
\local{Lagarto - SE}
\orientador{Nome-do-Orientador}
\coorientador{}
\instituicao{%
  Instituto Federal de Sergipe
  \par
  Sistemas de informação
}
\tipotrabalho{Monografia}

% O preambulo deve conter o tipo do trabalho, o objetivo (propósito),
% o nome da instituição e a área de concentração.
% Esse texto irá compor a Folha de Rosto e Folha de Aprovação.
\preambulo{
Trabalho de conclusão de curso apresentado ao Instituto Federal de
Sergipe como pré-requisito para a obtenção do grau de bacharel em
sistemas de informação.
\newline\textbf{Área de concentração}: Computação.
}%% fim do preambulo




% ---
% Configurações de aparência do PDF final

% alterando o aspecto da cor azul
\definecolor{blue}{RGB}{41,5,195}

% informações do PDF
\makeatletter
\hypersetup{
     	%pagebackref=true,
		pdftitle={\@title},
		pdfauthor={\@author},
    	pdfsubject={\imprimirpreambulo},
	    pdfcreator={LaTeX with abnTeX2 and Limarka},
		pdfkeywords={abnt}{latex}{abntex}{abntex2}{trabalho acadêmico}{limarka},
		colorlinks=false,       		% false: boxed links; true: colored links
    	linkcolor=black,          	% color of internal links
    	citecolor=black,        		% color of links to bibliography
    	filecolor=black,      		% color of file links
		urlcolor=black,
		bookmarksdepth=4
}
\makeatother
% ---

% ---
% Possibilita criação de Quadros e Lista de quadros.
% Ver https://github.com/abntex/abntex2/issues/176
%
\newcommand{\quadroname}{Quadro}
\newcommand{\listofquadrosname}{Lista de quadros}

\newfloat[chapter]{quadro}{loq}{\quadroname}
\newlistof{listofquadros}{loq}{\listofquadrosname}
\newlistentry{quadro}{loq}{0}

% configurações para atender às regras da ABNT
\setfloatadjustment{quadro}{\centering}
\counterwithout{quadro}{chapter}
\renewcommand{\cftquadroname}{\quadroname\space}
\renewcommand*{\cftquadroaftersnum}{\hfill--\hfill}

% ---


% ---
% Espaçamentos entre linhas e parágrafos
% ---

% O tamanho do parágrafo é dado por:
\setlength{\parindent}{1.3cm}

% Controle do espaçamento entre um parágrafo e outro:
\setlength{\parskip}{0.2cm}  % tente também \onelineskip

% ---
% compila o indice
% ---
\makeindex
% ---

%---
% CONFIGURAÇÕES EXTRA DO LIMARKA
%---

% Configura citações de pandoc para 4cm à esquerda (utiliza o ambiente quote)
\renewenvironment{quote}
  {\small\list{}{\rightmargin=0.1cm \leftmargin=4cm}%
   \item\relax}
  {\endlist}

% Para incluir páginas PDF (ficha catalografica e folha de aprovação)
\usepackage[dvipsnames]{xcolor} % http://tex.stackexchange.com/questions/124636/package-xcolor-error-undefined-colors-maroon-royal-blue-when-master-has-pdf
\usepackage{pdfpages}
\usepackage{longtable,ltcaption,booktabs} % para as tabelas pandoc e quadros ABNT
%\usepackage{floatrow}
%\floatsetup[figure]{capposition=top}

% Para melhorar o visual do quadro
\usepackage{boldline} 
\def\toprule{\hlineB{3}} % primeira linha mais gorda
\def\midrule{\hline}
\def\bottomrule{\hlineB{3}} % última linha mais gorda



% ---
% BUG: Imagens e tabelas apareciam no meio da página em branco
% https://github.com/abntex/trabalho-academico-limarka/issues/1
% O código a seguir posta imagens ou tabelas em página em branco no topo, em vez do meio (comportamento padrão)
\makeatletter
\setlength{\@fptop}{5pt} % Set distance from top of page to first float
\makeatother
% ---

% ---
% Usado pelo limarka como hook para criação de novas listas.
% https://github.com/abntex/trabalho-academico-limarka/issues/16
%
\newcommand{\listasdousuario}{}

% ---
% CUSTOMIZAÇÕES DO USUÁRIO (somente se existir arquivo config/latexcustomizacao.sty)
% ---
\IfFileExists{latexcustomizacao.sty}{\usepackage{latexcustomizacao}}{}

%%
%% Esse modelo é responsável pela impressão dos seguintes elementos:
%% Capa, Folha de rosto e Ficha catalográfica.

\special{dvipdfmx:config z 0}

% ----
% Início do documento
% ----
\begin{document}

% Seleciona o idioma do documento (conforme pacotes do babel)
%\selectlanguage{english}
\selectlanguage{brazil}

% Retira espaço extra obsoleto entre as frases.
\frenchspacing

% ----------------------------------------------------------
% ELEMENTOS PRÉ-TEXTUAIS
% ----------------------------------------------------------
% \pretextual

% ---
% Capa 
% ---
% Gerando capa abnTeX2
\imprimircapa

% ---

% ----------------------------------------------------------
% ELEMENTOS PRÉ-TEXTUAIS
% ----------------------------------------------------------
% \pretextual

% ---
% Folha de rosto: sempre será impressa
% ---
\imprimirfolhaderosto

% ---
% Sem ficha catalográfica
% ---
% ---


% ---
% ERRATA: Sem errata
% ---


% ---
% Sem Folha de aprovação
% ---
% ---
% ---
% Dedicatória
% ---
% ---
% Agradecimentos
% ---
\begin{agradecimentos}

Aos professores, pelas correções e ensinamentos que me permitiram
apresentar um melhor desempenho no meu processo de formação profissional
ao longo do curso.

\end{agradecimentos}
% ---
% ---
% Epígrafe
% ---
% ---

% ---
% Resumo na língua vernácula (obrigatório)
% ---


% resumo em português
\setlength{\absparsep}{18pt} % ajusta o espaçamento dos parágrafos do resumo
\begin{resumo}

  A programação é uma das áreas mais populares e importantes da tecnologia
  da informação. Por isso, muitos buscam uma formação em cursos nessa
  areá. No entanto, aprender a programar pode ser tornar uma tarefa
  difícil, estressante e frustrante para iniciantes. Isso se deve à
  necessidade de memorização e repetição.
  
  Nesse contexto, este artigo tem como objetivo desenvolver e avaliar um
  protótipo de uma plataforma de aprendizagem online que aproveita
  elementos da gamificação para incentivar os alunos a aprender e resolver
  problemas de lógica de programação por meio de missões desafiadoras. Os
  alunos ficam mais motivados, uma vez que suas ações resultariam em
  recompensas, conquistas e níveis que se correlacionam com seu
  desempenho.

 \textbf{Palavras-chave}: gamificação, aprendizagem de programação, eLearning, desenvolvimento
\end{resumo}


% ---
% Resumo em língua estrangeira (obrigatório)
% ---

% resumo em inglês
\begin{resumo}[Abstract]
 \begin{otherlanguage*}{english}
   Programming is one of the most popular and important areas of
   information technology. For this reason, many people seek an education
   in programming courses. However, learning to program can be a difficult,
   stressful and frustrating task for beginners. This is due to the need
   for memorization and repetition.
   
   In this context, this paper aims to develop and evaluate a prototype of
   an online learning platform that takes advantage of gamification
   elements to encourage students to learn and solve logic programming
   problems through challenging missions. Students are more motivated since
   their actions would result in rewards, achievements, and levels that
   correlate with their performance.

   \vspace{\onelineskip}
 
   \noindent 
   \textbf{Keywords}: gamification, programming learning, eLearning, development
 \end{otherlanguage*}
\end{resumo}


% ---

% ---
% Lista de ilustrações (opcional): não utilizando.
% ---


% ---
% Lista de quadros (opcional): não utilizando.
% ---

% Carrega listas definidas pelo usuário em `latexcustomizacao.sty`
\listasdousuario
% ---
% Lista de tabelas (opcional): não utilizando
% ---

% ---
% Lista de abreviaturas e siglas (opcional)
% ---
\begin{siglas}
  \item[ABNT] Associação Brasileira de Normas Técnicas
\end{siglas}
% ---

% ---
% Lista de símbolos (opcional): AUSENTE
% ---
% ---
% Sumário
% ---
\pdfbookmark[0]{\contentsname}{toc}
\tableofcontents*
\cleardoublepage
% ---


% ----------------------------------------------------------
% ELEMENTOS TEXTUAIS
% ----------------------------------------------------------
\textual
\pagestyle{simple}                  % #17 Cabeçalho apenas com
\aliaspagestyle{chapter}{simple}    % a numeração das páginas


\hypertarget{introduuxe7uxe3o}{%
\chapter{Introdução}\label{introduuxe7uxe3o}}

A programação é uma das áreas mais populares e importantes da tecnologia
da informação. Além disso, os conhecimentos de programação tornam-se
cada vez mais valiosos. À medida que as tecnologias se torna mais
avançada e difundida, a procura por programadores de software
especializados dispara. Segundo a \cite{procura-profissionais}, a
procura por profissionais de tecnologia cresceu mais de 670\% só em
2020. Um dos profissionais mais procurados é o desenvolvedor,
fundamental para o avanço da revolução tecnológica em curso.

No entanto, pode ser difícil aprender uma linguagem de programação, já
que o aprendizado pode ser desafiador, estressante e frustrante para
iniciantes. Isso se deve à complexidade da programação, à necessidade de
memorização e repetição e à dificuldade de encontrar a solução certa.

Além disso, aprender uma linguagem de programação é demorado e requer
perseverança para continuar praticando. Isso ocorre porque requer um
conhecimento de funções e sintaxe específicas sobre a linguagem. Também
é importante conhecer o hardware do computador e como os computadores
funcionam. Para se tornar um programador proficiente, você precisa
gastar muito tempo aprendendo sobre computadores e ciência da
computação.

Sem esse investimento de tempo, seria muito difícil ou impossível para
você se tornar um programador proficiente. Caso o aluno não tenha uma
motivação adequada, é provável ele desista. Neste contexto, elementos
semelhantes a jogos pode fornecer um ambiente motivador para o processo
de ensino de uma linguagem de programação.

\begin{quote}
A utilização dos jogos digitais online, com objetivos e metas
pré-estabelecidas em sua utilização, reforçam a ideia de atividades com
intenção de ensino e aprendizagem e não de lazer. Espera-se do professor
a busca por elementos que sejam adequados e facilitem o modo de aprender
da criança, utilizando a forma lúdica e atrativa que os jogos
pedagógicos oferecem aos alunos. \cite[p. 6]{utilizacao-jogos-digitais}
\end{quote}

Ao analisar o uso de mecanismos de gamificação no ensino de programação,
nota-se que há uma clara necessidade de analisar a criação de softwares
web projetados para usar elementos baseados em jogos para engajar os
alunos, estimular a ação, facilitar o aprendizado. Além disso, o sistema
deve incluir recursos como tarefas desafiadoras, níveis de
classificações, recompensas e conquistas.

Este artigo apresenta os resultados da pesquisa e desenvolvimento de um
sistema de E-Learning voltado ao ensino de linguagens de programação com
elementos de gamificação. A plataforma foi projetada para facilitar o
aprendizado de novas linguagens de programação.

\hypertarget{justificativa}{%
\section{Justificativa}\label{justificativa}}

\hypertarget{objetivo}{%
\section{Objetivo}\label{objetivo}}

\hypertarget{objetivos-especuxedficos}{%
\subsection{Objetivos Específicos}\label{objetivos-especuxedficos}}

\hypertarget{organizauxe7uxe3o-da-proposta}{%
\section{Organização da Proposta}\label{organizauxe7uxe3o-da-proposta}}

\hypertarget{fundamentauxe7uxe3o-teuxf3rica}{%
\chapter{Fundamentação Teórica}\label{fundamentauxe7uxe3o-teuxf3rica}}

\hypertarget{metodologia}{%
\chapter{Metodologia}\label{metodologia}}

\hypertarget{cronograma-de-atividades}{%
\chapter{Cronograma de Atividades}\label{cronograma-de-atividades}}

\hypertarget{atividades-realizadas}{%
\section{Atividades realizadas}\label{atividades-realizadas}}

\hypertarget{atividades-previstas}{%
\section{Atividades previstas}\label{atividades-previstas}}

\hypertarget{cronograma}{%
\section{Cronograma}\label{cronograma}}

% ----------------------------------------------------------
% ELEMENTOS PÓS-TEXTUAIS
% ----------------------------------------------------------
\postextual
% ----------------------------------------------------------

% ----------------------------------------------------------
% Início dos ELEMENTOS PÓS-TEXTUAIS
% ----------------------------------------------------------
\postextual
% ----------------------------------------------------------

% ----------------------------------------------------------
% Referências bibliográficas
% ----------------------------------------------------------
\bibliography{xxx-referencias}
% ----------------------------------------------------------
% Apêndices
% ----------------------------------------------------------
%% 
% Seção de apendices configurada como desativada
%% 
% ---

% ----------------------------------------------------------
% Anexos desativados: 
% Seção de anexos configurada como desativada
% ----------------------------------------------------------



\end{document}

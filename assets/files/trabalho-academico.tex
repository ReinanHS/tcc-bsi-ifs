%%%%%%%%%%%%%%%%%%%%%%%%%5
%% abtex2-modelo-trabalho-academico.tex, v<VERSION> laurocesar
%% Copyright 2012-<COPYRIGHT_YEAR> by abnTeX2 group at http://www.abntex.net.br/
%%
%% This work may be distributed and/or modified under the
%% conditions of the LaTeX Project Public License, either version 1.3
%% of this license or (at your option) any later version.
%% The latest version of this license is in
%%   http://www.latex-project.org/lppl.txt
%% and version 1.3 or later is part of all distributions of LaTeX
%% version 2005/12/01 or later.
%%
%% This work has the LPPL maintenance status `maintained'.
%%
%% The Current Maintainer of this work is the abnTeX2 team, led
%% by Lauro César Araujo. Further information are available on
%% http://www.abntex.net.br/
%%
%% This work consists of the files abntex2-modelo-trabalho-academico.tex,
%% abntex2-modelo-include-comandos and abntex2-modelo-references.bib
%%

% ------------------------------------------------------------------------
% ------------------------------------------------------------------------
% abnTeX2: Modelo de Trabalho Academico (tese de doutorado, dissertacao de
% mestrado e trabalhos monograficos em geral) em conformidade com
% ABNT NBR 14724:2011: Informacao e documentacao - Trabalhos academicos -
% Apresentacao
% ------------------------------------------------------------------------
% ------------------------------------------------------------------------

\documentclass[
	% -- opções da classe memoir --
	12pt,				% tamanho da fonte
%  openright,			% capítulos começam em pág ímpar (insere página vazia caso preciso)
	oneside,			% para impressão em recto e verso use twoside
	a4paper,			% tamanho do papel.
	% -- opções da classe abntex2 --
	%chapter=TITLE,		% títulos de capítulos convertidos em letras maiúsculas
	%section=TITLE,		% títulos de seções convertidos em letras maiúsculas
	%subsection=TITLE,	% títulos de subseções convertidos em letras maiúsculas
	%subsubsection=TITLE,% títulos de subsubseções convertidos em letras maiúsculas
	% -- opções do pacote babel --
	english,			% idioma adicional para hifenização
	french,				% idioma adicional para hifenização
	spanish,			% idioma adicional para hifenização
	brazil				% o último idioma é o principal do documento
	]{abntex2}

% ---
% Pacotes básicos
% ---
\usepackage{times}			    % Usa fonte times
\renewcommand{\ABNTEXchapterfont}{\normalfont} % para aplicar a fonte escolhida em tudo
\usepackage[T1]{fontenc}		% Selecao de codigos de fonte.
\usepackage[utf8]{inputenc}		% Codificacao do documento (conversão automática dos acentos)
\usepackage{lastpage}			% Usado pela Ficha catalográfica
\usepackage{indentfirst}		% Indenta o primeiro parágrafo de cada seção.
\usepackage{color}				% Controle das cores
\usepackage{graphicx}			% Inclusão de gráficos
\usepackage{microtype} 			% para melhorias de justificação
% ---

% ---
% Pacotes adicionais, usados apenas no âmbito do Modelo Canônico do abnteX2
% ---
\usepackage{lipsum}				% para geração de dummy text
% ---

% ---
% Pacotes de citações
% ---
\usepackage[brazilian,hyperpageref]{backref}	 % Paginas com as citações na bibl
\usepackage[alf]{abntex2cite}	% Citações padrão ABNT

% ---
% CONFIGURAÇÕES DE PACOTES
% ---

% ---
% Configurações do pacote backref
% Usado sem a opção hyperpageref de backref
\renewcommand{\backrefpagesname}{Citado na(s) página(s):~}
% Texto padrão antes do número das páginas
\renewcommand{\backref}{}
% Define os textos da citação
\renewcommand*{\backrefalt}[4]{
	\ifcase #1 %
		Nenhuma citação no texto.%
	\or
		Citado na página #2.%
	\else
		Citado #1 vezes nas páginas #2.%
	\fi}%
% ---

% ---
% Informações de dados para CAPA e FOLHA DE ROSTO
% ---
\titulo{Aprimoramento da ferramenta Limarka e documentando seu uso efetivo:
Facilitando a elaboração de TCCs em Markdown}
\autor{Reinan Gabriel Dos Santos Souza}
\data{2023}
\local{Lagarto - SE}
\orientador{Nome-do-Orientador}
\coorientador{}
\instituicao{%
  Instituto Federal de Sergipe
  \par
  Sistemas de informação
}
\tipotrabalho{Monografia}

% O preambulo deve conter o tipo do trabalho, o objetivo (propósito),
% o nome da instituição e a área de concentração.
% Esse texto irá compor a Folha de Rosto e Folha de Aprovação.
\preambulo{
Trabalho de Conclusão de Curso apresentado ao Curso de Graduação em
Sistemas de Informação do Campus Lagarto do Instituto Federal de
Educação, Ciência e Tecnologia, como requisito parcial à obtenção do
grau de bacharel em Sistemas de Informação.
\newline\textbf{Área de concentração}: Computação.
}%% fim do preambulo




% ---
% Configurações de aparência do PDF final

% alterando o aspecto da cor azul
\definecolor{blue}{RGB}{41,5,195}

% informações do PDF
\makeatletter
\hypersetup{
     	%pagebackref=true,
		pdftitle={\@title},
		pdfauthor={\@author},
    	pdfsubject={\imprimirpreambulo},
	    pdfcreator={LaTeX with abnTeX2 and Limarka},
		pdfkeywords={abnt}{latex}{abntex}{abntex2}{trabalho acadêmico}{limarka},
		colorlinks=false,       		% false: boxed links; true: colored links
    	linkcolor=black,          	% color of internal links
    	citecolor=black,        		% color of links to bibliography
    	filecolor=black,      		% color of file links
		urlcolor=black,
		bookmarksdepth=4
}
\makeatother
% ---

% ---
% Possibilita criação de Quadros e Lista de quadros.
% Ver https://github.com/abntex/abntex2/issues/176
%
\newcommand{\quadroname}{Quadro}
\newcommand{\listofquadrosname}{Lista de quadros}

\newfloat[chapter]{quadro}{loq}{\quadroname}
\newlistof{listofquadros}{loq}{\listofquadrosname}
\newlistentry{quadro}{loq}{0}

% configurações para atender às regras da ABNT
\setfloatadjustment{quadro}{\centering}
\counterwithout{quadro}{chapter}
\renewcommand{\cftquadroname}{\quadroname\space}
\renewcommand*{\cftquadroaftersnum}{\hfill--\hfill}

% ---


% ---
% Espaçamentos entre linhas e parágrafos
% ---

% O tamanho do parágrafo é dado por:
\setlength{\parindent}{1.3cm}

% Controle do espaçamento entre um parágrafo e outro:
\setlength{\parskip}{0.2cm}  % tente também \onelineskip

% ---
% compila o indice
% ---
\makeindex
% ---

%---
% CONFIGURAÇÕES EXTRA DO LIMARKA
%---

% Configura citações de pandoc para 4cm à esquerda (utiliza o ambiente quote)
\renewenvironment{quote}
  {\small\list{}{\rightmargin=0.1cm \leftmargin=4cm}%
   \item\relax}
  {\endlist}

% Para incluir páginas PDF (ficha catalografica e folha de aprovação)
\usepackage[dvipsnames]{xcolor} % http://tex.stackexchange.com/questions/124636/package-xcolor-error-undefined-colors-maroon-royal-blue-when-master-has-pdf
\usepackage{pdfpages}
\usepackage{longtable,ltcaption,booktabs} % para as tabelas pandoc e quadros ABNT
%\usepackage{floatrow}
%\floatsetup[figure]{capposition=top}

% Para melhorar o visual do quadro
\usepackage{boldline} 
\def\toprule{\hlineB{3}} % primeira linha mais gorda
\def\midrule{\hline}
\def\bottomrule{\hlineB{3}} % última linha mais gorda



% ---
% BUG: Imagens e tabelas apareciam no meio da página em branco
% https://github.com/abntex/trabalho-academico-limarka/issues/1
% O código a seguir posta imagens ou tabelas em página em branco no topo, em vez do meio (comportamento padrão)
\makeatletter
\setlength{\@fptop}{5pt} % Set distance from top of page to first float
\makeatother
% ---

% ---
% Usado pelo limarka como hook para criação de novas listas.
% https://github.com/abntex/trabalho-academico-limarka/issues/16
%
\newcommand{\listasdousuario}{}

% ---
% CUSTOMIZAÇÕES DO USUÁRIO (somente se existir arquivo config/latexcustomizacao.sty)
% ---
\IfFileExists{latexcustomizacao.sty}{\usepackage{latexcustomizacao}}{}

%%
%% Esse modelo é responsável pela impressão dos seguintes elementos:
%% Capa, Folha de rosto e Ficha catalográfica.

\special{dvipdfmx:config z 0}

% ----
% Início do documento
% ----
\begin{document}

% Seleciona o idioma do documento (conforme pacotes do babel)
%\selectlanguage{english}
\selectlanguage{brazil}

% Retira espaço extra obsoleto entre as frases.
\frenchspacing

% ----------------------------------------------------------
% ELEMENTOS PRÉ-TEXTUAIS
% ----------------------------------------------------------
% \pretextual

% ---
% Capa 
% ---
% Gerando capa abnTeX2
\imprimircapa

% ---

% ----------------------------------------------------------
% ELEMENTOS PRÉ-TEXTUAIS
% ----------------------------------------------------------
% \pretextual

% ---
% Folha de rosto: sempre será impressa
% ---
\imprimirfolhaderosto

% ---
% Sem ficha catalográfica
% ---
% ---


% ---
% ERRATA: Sem errata
% ---


% ---
% Sem Folha de aprovação
% ---
% ---
% ---
% Dedicatória
% ---
% ---
% Agradecimentos
% ---
\begin{agradecimentos}

Gostaria de expressar meus mais sinceros agradecimentos aos estimados
professores do Instituto Federal de Educação, Ciência e Tecnologia de
Sergipe (IFS). Este trabalho de conclusão de curso não teria sido
possível sem a orientação, o apoio e a sabedoria que vocês generosamente
compartilharam ao longo da minha jornada acadêmica.

Suas aulas inspiradoras, seu compromisso com a excelência educacional e
seu incentivo constante foram fundamentais para o meu desenvolvimento
acadêmico e pessoal. Cada um de vocês desempenhou um papel fundamental
no meu crescimento como estudante e na minha capacidade de enfrentar os
desafios acadêmicos.

Este trabalho é uma celebração do aprendizado e da dedicação que
testemunhei no IFS. Agradeço sinceramente a todos os professores que
fizeram parte desta jornada, contribuindo para o meu crescimento
acadêmico e pessoal.

\end{agradecimentos}
% ---
% ---
% Epígrafe
% ---
% ---

% ---
% Resumo na língua vernácula (obrigatório)
% ---


% resumo em português
\setlength{\absparsep}{18pt} % ajusta o espaçamento dos parágrafos do resumo
\begin{resumo}

  Este trabalho de pesquisa concentra-se na aprimoração da ferramenta
  Limarka, destinada à elaboração de Trabalhos de Conclusão de Curso
  (TCCs) em formato Markdown. Além disso, o estudo aborda a criação de
  documentação abrangente que visa destacar e facilitar o uso efetivo
  dessa ferramenta. O objetivo principal é melhorar a experiência dos
  estudantes na elaboração de TCCs, proporcionando uma ferramenta mais
  eficaz e fornecendo orientações detalhadas por meio da documentação. Ao
  unir esses esforços de aprimoramento e documentação, busca-se
  simplificar o processo de criação de TCCs em Markdown, tornando-o mais
  acessível e eficiente para os acadêmicos.

 \textbf{Palavras-chave}: Limarka, Markdown, Melhoria, TCCs, Ferramenta, Aprimoramento e
Produtividade
\end{resumo}


% ---
% Resumo em língua estrangeira (obrigatório)
% ---

% resumo em inglês
\begin{resumo}[Abstract]
 \begin{otherlanguage*}{english}
   This research work focuses on improving the Limarka tool, designed for
   preparing Course Completion Papers (TCCs) in Markdown format.
   Furthermore, the study addresses the creation of comprehensive
   documentation that aims to highlight and facilitate the effective use of
   this tool. The main objective is to improve students' experience in
   preparing TCCs, providing a more effective tool and providing detailed
   guidance through documentation. By combining these improvement and
   documentation efforts, we seek to simplify the process of creating TCCs
   in Markdown, making it more accessible and efficient for academics.

   \vspace{\onelineskip}
 
   \noindent 
   \textbf{Keywords}: Limarka, Markdown, Improvement, TCCs, Tool, Improvement and Productivity
 \end{otherlanguage*}
\end{resumo}


% ---

% ---
% Lista de ilustrações (opcional): não utilizando.
% ---


% ---
% Lista de quadros (opcional): não utilizando.
% ---

% Carrega listas definidas pelo usuário em `latexcustomizacao.sty`
\listasdousuario
% ---
% Lista de tabelas (opcional): não utilizando
% ---

% ---
% Lista de abreviaturas e siglas (opcional)
% ---
\begin{siglas}
  \item[ABNT] Associação Brasileira de Normas Técnicas
  \item[TCC] Trabalho de Conclusão de Curso
  \item[BSI] Bacharelado em Sistemas de Informação
\end{siglas}
% ---

% ---
% Lista de símbolos (opcional): AUSENTE
% ---
% ---
% Sumário
% ---
\pdfbookmark[0]{\contentsname}{toc}
\tableofcontents*
\cleardoublepage
% ---


% ----------------------------------------------------------
% ELEMENTOS TEXTUAIS
% ----------------------------------------------------------
\textual
\pagestyle{simple}                  % #17 Cabeçalho apenas com
\aliaspagestyle{chapter}{simple}    % a numeração das páginas


\hypertarget{introduuxe7uxe3o}{%
\chapter{Introdução}\label{introduuxe7uxe3o}}

\hypertarget{justificativa}{%
\section{Justificativa}\label{justificativa}}

A elaboração de trabalhos de conclusão de curso (TCCs) representa uma
etapa fundamental na jornada acadêmica dos estudantes de graduação,
exigindo a aplicação de conhecimentos adquiridos ao longo do curso e a
produção de um documento formal que atenda a rigorosos padrões
acadêmicos. Neste contexto, a ferramenta Limarka emerge como uma solução
inovadora para a elaboração de TCCs em formato Markdown, prometendo uma
abordagem mais simplificada e eficiente em comparação com os editores de
texto tradicionais. No entanto, a adoção da ferramenta Limarka ainda
enfrenta desafios significativos, principalmente devido à falta de
documentação abrangente e suporte para as necessidades específicas dos
estudantes do curso de Bacharelado em Sistemas de Informação do
Instituto Federal de Sergipe (IFS), Campus Lagarto.

Este trabalho justifica-se pela necessidade premente de superar esses
obstáculos, aprimorando a ferramenta Limarka para torná-la mais
acessível, intuitiva e adaptada às especificidades dos padrões
acadêmicos brasileiros, em especial às normas da ABNT. Além disso, a
elaboração de documentação detalhada e orientada ao usuário visa equipar
os estudantes com os recursos necessários para aproveitar plenamente os
benefícios da escrita em Markdown, otimizando o processo de elaboração
de TCCs.

A importância deste estudo é evidenciada pelo potencial de transformar
significativamente a experiência de escrita acadêmica dos estudantes,
permitindo uma maior concentração no conteúdo científico em detrimento
das preocupações com a formatação do documento. Ao facilitar o acesso e
a utilização da ferramenta Limarka, espera-se não apenas melhorar a
eficiência e a qualidade dos trabalhos produzidos, mas também incentivar
a adoção de práticas de escrita colaborativa e a utilização de
tecnologias abertas e acessíveis no ambiente acadêmico.

A contribuição deste trabalho estende-se para além dos limites do IFS,
Campus Lagarto, uma vez que as melhorias implementadas na ferramenta
Limarka e a documentação produzida podem servir como referência para
outras instituições de ensino superior que buscam inovar nas
metodologias de elaboração de TCCs. Desta forma, este TCC não apenas
atende a uma demanda específica do curso de Bacharelado em Sistemas de
Informação, mas também contribui para o avanço das práticas de escrita
científica no contexto acadêmico brasileiro, promovendo a inclusão
digital e a democratização do acesso a ferramentas educacionais de
qualidade.

Portanto, o aprimoramento da ferramenta Limarka e a criação de uma
documentação detalhada e acessível representam passos importantes na
direção de um ensino superior mais inovador, eficiente e inclusivo,
alinhado às demandas contemporâneas por tecnologias educacionais que
facilitam a produção de conhecimento científico e sua disseminação.

\hypertarget{objetivo}{%
\section{Objetivo}\label{objetivo}}

Nesta seção, serão é apresentado o objetivo geral e os objetivos
específicos desta pesquisa.

\hypertarget{objetivo-geral}{%
\subsection{Objetivo Geral}\label{objetivo-geral}}

O objetivo geral deste trabalho é construir um ecossistema, com manual
de instruções, para a ferramenta Lamarka que permita a escrita de
trabalhos cientificos de forma mais acessível e eficaz pelos os
estudantes do curso de Bacharelado em Sistemas de Informação (BSI) do
IFS Campus Lagarto.

\hypertarget{objetivos-especuxedficos}{%
\subsection{Objetivos Específicos}\label{objetivos-especuxedficos}}

A fim de atingir o objetivo geral, são definidos os seguintes objetivos
específicos:

\begin{itemize}
\tightlist
\item
  \textbf{Disponibilizar o Lamarka por meio de um ambiente Docker}:
  Facilitar o acesso e a instalação da ferramenta, tornando-a disponível
  em um ambiente Docker de fácil configuração;
\item
  \textbf{Desenvolver uma ferramenta de linha de comando}: Criar
  comandos que simplifiquem o processo de construção de documentos em
  Markdown usando o Lamarka;
\item
  \textbf{Implementar um pipeline no Github Actions}: Estabelecer uma
  estrutura de pipeline automatizado para compilar projetos Lamarka de
  maneira eficiente no GitHub;
\item
  \textbf{Habilitar exportação para HTML no Github Pages}: Aprimorar a
  capacidade do Lamarka de exportar documentos em Markdown para o
  formato HTML, tornando-os acessíveis no Github Pages;
\item
  \textbf{Integrar funcionalidade de importação de arquivos markdown}:
  Adicionar a capacidade de importar documentos Markdown existentes,
  simplificando o processo de compilação do Lamarka;
\item
  \textbf{Reestruturar a organização de arquivos do template}: Melhorar
  a estrutura de arquivos do template para torná-lo mais intuitivo e
  adequado aos padrões do IFS Campus Lagarto;
\item
  \textbf{Conduzir testes rigorosos}: Realizar testes rigorosos para
  garantir que as melhorias no Lamarka atendam às necessidades dos
  estudantes, garantindo sua eficácia;
\item
  \textbf{Produzir uma documentação abrangente}: Criar uma documentação
  detalhada, abrangendo desde a instalação até a formatação de
  documentos em conformidade com os padrões do IFS Campus Lagarto;
\item
  \textbf{Promover a adoção e divulgação}: Incentivar ativamente a
  utilização da ferramenta aprimorada e da documentação entre os
  estudantes do curso de BSI, buscando uma ampla adoção do Lamarka como
  ferramenta eficaz para a elaboração de TCCs no IFS;
\item
  \textbf{Avaliar o impacto das melhorias}: Realizar uma análise crítica
  para avaliar o impacto das melhorias no Lamarka na eficiência e na
  qualidade da produção de TCCs, coletando feedback dos usuários e
  ajustando a ferramenta conforme necessário.
\end{itemize}

\hypertarget{organizauxe7uxe3o-da-proposta}{%
\section{Organização da Proposta}\label{organizauxe7uxe3o-da-proposta}}

\hypertarget{fundamentauxe7uxe3o-teuxf3rica}{%
\chapter{Fundamentação Teórica}\label{fundamentauxe7uxe3o-teuxf3rica}}

\hypertarget{metodologia}{%
\section{Metodologia}\label{metodologia}}

\hypertarget{abordagem-da-pesquisa}{%
\subsection{Abordagem da pesquisa}\label{abordagem-da-pesquisa}}

Este trabalho adota uma abordagem de pesquisa aplicada, focada no
desenvolvimento e na avaliação de uma solução tecnológica destinada a
facilitar a elaboração de trabalhos de conclusão de curso (TCCs) em
formato Markdown. O objetivo é desenvolver melhorias na ferramenta
Limarka, assim como documentar seu uso de forma efetiva, visando
otimizar a experiência dos usuários, principalmente estudantes do curso
de Bacharelado em Sistemas de Informação. A metodologia é composta por
etapas de planejamento, desenvolvimento, teste e avaliação, seguindo
princípios da engenharia de software.

\hypertarget{desenvolvimento-da-ferramenta}{%
\subsection{Desenvolvimento da
ferramenta}\label{desenvolvimento-da-ferramenta}}

Para atender aos objetivos específicos propostos, o desenvolvimento da
ferramenta será realizado em etapas, iniciando com a configuração de um
ambiente Docker para facilitar o acesso e a instalação da ferramenta
Limarka. Em seguida, será desenvolvida uma interface de linha de comando
para simplificar a criação e a compilação de documentos Markdown. A
implementação de um pipeline automatizado no GitHub Actions será
realizada para permitir a compilação eficiente de projetos Limarka. Além
disso, serão implementadas funcionalidades para exportação de documentos
para HTML via GitHub Pages e para importação de arquivos Markdown
existentes.

\hypertarget{estratuxe9gias-de-teste}{%
\subsection{Estratégias de teste}\label{estratuxe9gias-de-teste}}

Para garantir a qualidade e a eficácia das melhorias implementadas,
serão conduzidos testes rigorosos em cada etapa do desenvolvimento.
Esses testes incluirão testes unitários para validar funcionalidades
individuais, testes de integração para assegurar que os componentes da
ferramenta trabalhem conjuntamente de forma eficiente e testes de
usabilidade para avaliar a experiência do usuário.

\hypertarget{documentauxe7uxe3o}{%
\subsection{Documentação}\label{documentauxe7uxe3o}}

Uma parte crucial deste trabalho é a produção de documentação
abrangente, que cobrirá desde a instalação da ferramenta e configuração
do ambiente até a utilização efetiva da mesma para a elaboração de TCCs.
A documentação será estruturada de maneira clara e acessível, visando
facilitar o uso da ferramenta por parte dos estudantes e docentes.

\hypertarget{avaliauxe7uxe3o-e-feedback}{%
\subsection{Avaliação e feedback}\label{avaliauxe7uxe3o-e-feedback}}

Após a implementação das melhorias e da documentação, será realizada uma
avaliação do impacto dessas mudanças na eficiência e qualidade da
elaboração de TCCs. Essa avaliação contará com a coleta de feedback de
usuários reais da ferramenta, incluindo estudantes e professores. Os
resultados dessa avaliação serão utilizados para realizar ajustes finais
na ferramenta e na documentação, garantindo que as necessidades dos
usuários sejam plenamente atendidas.

\hypertarget{cronograma-de-atividades}{%
\chapter{Cronograma de Atividades}\label{cronograma-de-atividades}}

\hypertarget{atividades-realizadas}{%
\section{Atividades realizadas}\label{atividades-realizadas}}

\hypertarget{atividades-previstas}{%
\section{Atividades previstas}\label{atividades-previstas}}

\hypertarget{cronograma}{%
\section{Cronograma}\label{cronograma}}

% ----------------------------------------------------------
% ELEMENTOS PÓS-TEXTUAIS
% ----------------------------------------------------------
\postextual
% ----------------------------------------------------------

% ----------------------------------------------------------
% Início dos ELEMENTOS PÓS-TEXTUAIS
% ----------------------------------------------------------
\postextual
% ----------------------------------------------------------

% ----------------------------------------------------------
% Referências bibliográficas
% ----------------------------------------------------------
\bibliography{xxx-referencias}
% ----------------------------------------------------------
% Apêndices
% ----------------------------------------------------------
%% 
% Seção de apendices configurada como desativada
%% 
% ---

% ----------------------------------------------------------
% Anexos desativados: 
% Seção de anexos configurada como desativada
% ----------------------------------------------------------



\end{document}
